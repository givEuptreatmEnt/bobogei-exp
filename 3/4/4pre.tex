\documentclass[12pt, a4paper]{article}

\usepackage[hmargin=2.5cm, vmargin=2cm]{geometry}
\usepackage{amsthm, amssymb, mathtools, yhmath, graphicx}
\usepackage{fontspec, type1cm, titlesec, titling, fancyhdr, tabularx}
\usepackage{caption}
\usepackage{color}
\usepackage{hhline}
\usepackage{unicode-math}
\usepackage{nicefrac}
\usepackage[abbreviations, per-mode=symbol]{siunitx}
\usepackage{comment}
\usepackage{float}
\usepackage{subcaption}

\usepackage[CheckSingle, CJKmath]{xeCJK}
\usepackage{CJKulem}
\usepackage{enumitem}
\usepackage[usenames, dvipsnames]{xcolor}
\usepackage{colortbl}
\usepackage{circuitikz}
%\setCJKmainfont[BoldFont=cwTex Q Hei]{cwTex Q Ming}
%\setCJKsansfont[BoldFont=cwTex Q Hei]{cwTex Q Ming}
%\setCJKmonofont[BoldFont=cwTex Q Hei]{cwTex Q Ming}
\setCJKmainfont[BoldFont=cwTeX Q Hei]{cwTeX Q Ming}

\def\normalsize{\fontsize{12}{18}\selectfont}
\def\large{\fontsize{14}{21}\selectfont}
\def\Large{\fontsize{16}{24}\selectfont}
\def\LARGE{\fontsize{18}{27}\selectfont}
\def\Huge{\fontsize{20}{30}\selectfont}

%\titleformat{\section}{\bf\Large}{\arabic{section}}{24pt}{}
%\titleformat{\subsection}{\large}{\\arabic{subsection}.}{12pt}{}
\titlespacing*{\subsection}{0pt}{0pt}{1.5ex}

\parindent=24pt

\DeclarePairedDelimiter{\abs}{\lvert}{\rvert}
\DeclarePairedDelimiter{\norm}{\lVert}{\rVert}
\DeclarePairedDelimiter{\inpd}{\langle}{\rangle}
\DeclarePairedDelimiter{\ceil}{\lceil}{\rceil}
\DeclarePairedDelimiter{\floor}{\lfloor}{\rfloor}

\newcommand{\unit}[1]{\:(\text{#1})}
\newcommand{\img}{\mathsf{i}}
\newcommand{\ex}{\mathsf{e}}
\newcommand{\dD}{\mathrm{d}}
\newcommand{\dI}{\,\mathrm{d}}
\DeclareSIUnit \uF {\micro \farad}
\DeclareSIUnit \mH {\milli \henry}

\newcommand{\tri}{$\rhd$}

\title{ \bf {\huge 電子電路實驗5: Multi-Pole Feedback Network OP-Amp Circuit}\\ 實驗預報}
\author{B02901178 江誠敏}
%\date{2014/09/21}

\begin{document}

\maketitle

\section{Objectives}
\begin{enumerate}
  \item To analyze the theory of feedback network in the multi-pole OP-Amp circuit.
  \item To discuss the issue of stability for the feedback amplifier.
  \item To understand the physical meaning of sinusoidal vibration.
\end{enumerate}


\section{Procedures}
\subsection{DC Functional Confirmation of $A_1$}
\begin{enumerate}[itemsep=0pt]
  \item Reference pin voltage for $A_1$, check $V_{pin7}, V_{pin4}, V_{pin2}, V_{pin3}, V_{pin6}$.
  \item Record these values.
\end{enumerate}

\subsection{DC Functional Confirmation $A_2$}
\begin{enumerate}[itemsep=0pt]
  \item Use $R = \SI{10}\kohm, r = \SI{100}\ohm, C_1 = \SI{0.1}\uF$ for $A_2$.
  \item Supply voltage source $V_{CC} = +\SI{15}\V \text{, and } -V_{CC} = \SI{-15}\V$ to the
    circuit.
  \item Reference pin voltage for $A_2$, check $V_{pin7}, V_{pin4}, V_{pin2}, V_{pin3}, V_{pin6}$.
  \item Record these values.
\end{enumerate}

\subsection{Small Signal Analysis}
\begin{enumerate}[itemsep=0pt]
  \item Use $R = \SI{10}\kohm, r = \SI{100}\ohm, C_1 = \SI{0.1}\uF$ for $A_2$.
  \item Supply voltage source $V_{CC} = +\SI{15}\V \text{, and } -V_{CC} = \SI{-15}\V$ to the
    circuit.
  \item Apply the input small signal $V_i$ to the breadboard by using function generator
    to generate $v_i = v_{ac} \sin(2 \pi f t), 2 v_{ac} = \SI{100}\mV_{(p-p)}, f = \SI{100}\Hz$.
  \item Make sure that the $v_i$ is measured from the breadboard by using the probe from $\texttt{CH1}$
    in oscilloscope.
  \item oscilloscope \tri Press the \texttt{CH1} and \texttt{CH2} \texttt{MENU} \tri \texttt{Coupling}
    \tri \texttt{AC}.
  \item Observe $V_{i(p-p)}$ and $V_{o(p-p)}$ in \texttt{CH1} and \texttt{CH2}, respectively.
  \item Keep the previous adjustment of $V_i$ constantly.
  \item Record the voltage gain $A_M$ in the oscilloscope.
  \item Function generator \tri Adjust Frequency and observe the voltage gain $A_V$ in
    oscilloscope until $A_v = 0.707 A_M$.
  \item Record the frequency $f_{3dB}$.
\end{enumerate}

\subsection{DC Functional Confirmation of $A_3$}
\begin{enumerate}[itemsep=0pt]
  \item Short terminal \texttt{D} to the ground.
  \item Reference pin voltage for $A_3$, check $V_{pin7}, V_{pin4}, V_{pin2}, V_{pin3}, V_{pin6}$.
  \item Record these values.
\end{enumerate}

\subsection{DC Functional Confirmation of $A_4$}
\begin{enumerate}[itemsep=0pt]
  \item Short terminal \texttt{E} to the ground.
  \item Reference pin voltage for $A_4$, check $V_{pin7}, V_{pin4}, V_{pin2}, V_{pin3}, V_{pin6}$.
  \item Record these values.
\end{enumerate}

\subsection{Initial state of the feedback network circuit}
\begin{enumerate}[itemsep=0pt]
  \item Use $R = \SI{10}\kohm, r = \SI{100}\ohm, C_1 = C_2 = C_3 \SI{0.1}\uF, VR = \SI{10}\kohm$.
  \item adjust $VR$ to have $R_{p1} = \SI{0}\ohm, R_{p2} = \SI{10}\kohm$.
  \item Apply the input signal $v_i$ to the breadboard by using function generator to generate
    $v_i = v_{ac} \operatorname{\text{square}}(2 \pi f t), 2 v_{ac} = \SI{5}\V_{(p-p)}, f = 0 \textasciitilde 10 \si{\Hz}$.
    circuit.
  \item Make sure that the $v_i$ is {\bf measured from the breadboard} by using the probe
    from \texttt{CH1} in oscilloscope.
  \item Oscilloscope \tri Press the \texttt{CH1} and \texttt{CH2} \texttt{MENU} \tri \texttt{Coupling} \tri \texttt{DC}.
  \item Observe whether the waveform shown in \texttt{CH1} and \texttt{CH2} distort.
\end{enumerate}

\subsection{Vibration observation of the circuit}
\begin{enumerate}[itemsep=0pt]
  \item Keep the previous adjustment in step 7 constantly.
  \item Observe the waveform of $V_{o(p-p)}$ in \texttt{CH2} when slowly increasing the value
    of $R_{p1}$ until the sinusoidal vibration occur.
  \item As the sinusoidal vibration occur, record $V_{S(p-p)},V_{J(p-p)},V_{o(p-p)}, f_o, R_{p1}, R_{p2}$.
  \item During the adjustment of appearing sin-vibration, observe whether the 
    waveform of $V_{o(p-p)}$ occur damping phenomenon.
\end{enumerate}
\end{document}


