\documentclass[12pt, a4paper]{article}

\usepackage[hmargin=2.5cm, vmargin=2cm]{geometry}
\usepackage{amsthm, amssymb, mathtools, yhmath, graphicx}
\usepackage{fontspec, type1cm, titlesec, titling, fancyhdr, tabularx}
\usepackage{caption}
\usepackage{color}
\usepackage{hhline}
\usepackage{unicode-math}
\usepackage{nicefrac}
\usepackage[abbreviations, per-mode=symbol]{siunitx}
\usepackage{comment}
\usepackage{float}
\usepackage{subcaption}

\usepackage[CheckSingle, CJKmath]{xeCJK}
\usepackage{CJKulem}
\usepackage{enumitem}
\usepackage[usenames, dvipsnames]{xcolor}
\usepackage{colortbl}
\usepackage{circuitikz}
%\setCJKmainfont[BoldFont=cwTex Q Hei]{cwTex Q Ming}
%\setCJKsansfont[BoldFont=cwTex Q Hei]{cwTex Q Ming}
%\setCJKmonofont[BoldFont=cwTex Q Hei]{cwTex Q Ming}
\setCJKmainfont[BoldFont=cwTeX Q Hei]{cwTeX Q Ming}

\def\normalsize{\fontsize{12}{18}\selectfont}
\def\large{\fontsize{14}{21}\selectfont}
\def\Large{\fontsize{16}{24}\selectfont}
\def\LARGE{\fontsize{18}{27}\selectfont}
\def\Huge{\fontsize{20}{30}\selectfont}

%\titleformat{\section}{\bf\Large}{\arabic{section}}{24pt}{}
%\titleformat{\subsection}{\large}{\\arabic{subsection}.}{12pt}{}
\titlespacing*{\subsection}{0pt}{0pt}{1.5ex}

\parindent=24pt

\DeclarePairedDelimiter{\abs}{\lvert}{\rvert}
\DeclarePairedDelimiter{\norm}{\lVert}{\rVert}
\DeclarePairedDelimiter{\inpd}{\langle}{\rangle}
\DeclarePairedDelimiter{\ceil}{\lceil}{\rceil}
\DeclarePairedDelimiter{\floor}{\lfloor}{\rfloor}

\newcommand{\unit}[1]{\:(\text{#1})}
\newcommand{\img}{\mathsf{i}}
\newcommand{\ex}{\mathsf{e}}
\newcommand{\dD}{\mathrm{d}}
\newcommand{\dI}{\,\mathrm{d}}
\DeclareSIUnit \uF {\micro \farad}
\DeclareSIUnit \mH {\milli \henry}

\newcommand{\tri}{$\rhd$}

\title{ \bf {\huge 電子電路實驗5: Oscillators}\\ 實驗預報}
\author{B02901178 江誠敏}
%\date{2014/09/21}

\begin{document}

\maketitle

\section{Objectives}
\begin{enumerate}
  \item To be familiar with the Barkhausen criterion
     and various kinds of oscillators.   
\end{enumerate}


\section{Procedures}
\subsection{Sinusoidal oscillators: Wien-bridge oscillator}
\begin{enumerate}[itemsep=0pt]
  \item In Fig. 3, disconnect $D_1, D_2, R_3, R_4, R_5, R_6$,
    use $R_1 = \SI{3.3}\kohm, R_2 = \SI{10}\kohm$.
  \item Oscilloscope \tri Press the \texttt{Display} button \tri 
    \texttt{Format} \tri \texttt{YT mode}.
  \item Measure and record $V_{out(p-p)}, f$.
  \item Now connect $D_1, D_2, R_3, R_4, R_5, R_6$,
    use $R_1 = \SI{3.3}\kohm, R_2 = \SI{10}\kohm$.
  \item Oscilloscope \tri Press the \texttt{Display} button \tri 
    \texttt{Format} \tri \texttt{YT mode}.
  \item Adjust VR($R_2$) until the sinusoidal vibration with the
    {\bf mininmum amplitude} occurs in $V_{out}$.
  \item Measure and record $V_{out(p-p)}, f, R_2, R_2/R_1$.
  \item Adjust VR($R_2$) until the sinusoidal vibration with the
    {\bf maximum amplitude} occurs in $V_{out}$.
  \item Measure and record $V_{out(p-p)}, f, R_2, R_2/R_1$.
\end{enumerate}

\subsection{Sinusoidal oscillators: Phase-shift oscillator}
\begin{enumerate}[itemsep=0pt]
  \item In Fig. 4, use the designed value from Pre-Lab Work for components.
  \item Oscilloscope \tri Press the \texttt{Display} button \tri Format \tri \texttt{YT mode}.

  \item Adjust VR($R_1$) until the sinusoidal vibration with the
    {\bf mininmum amplitude} occurs in $V_{out}$.
  \item Measure and record $V_{out(p-p)}, f, R_1, R_1/R1$.
  \item Adjust VR($R_1$) until the sinusoidal vibration with the
    {\bf maximum amplitude} occurs in $V_{out}$.
  \item Measure and record $V_{out(p-p)}, f, R_1, R_1/R$.

\end{enumerate}

\subsection{Triangular waveform generator: RC-circuit bistable multivibrator}
\begin{enumerate}[itemsep=0pt]
  \item In Fig. 5, use the designed value from PR for components.
  \item Oscilloscope \tri Press the \texttt{Display} button \tri \texttt{Format} 
    \tri \texttt{YT mode}.
  \item Record $V_{out(p-p)}, f, R_1, R_2, R_3, C$
\end{enumerate}

\subsection{Triangular waveform generator: Integrator bistable multivibrator}
\begin{enumerate}[itemsep=0pt]
  \item In Fig. 6, use the designed value from PR for components.
  \item Oscilloscope \tri Press the \texttt{Display} button \tri \texttt{Format} 
    \tri \texttt{YT mode}.
  \item Record $V_{out(p-p)}, f, R_2, R_3$
\end{enumerate}

\end{document}


