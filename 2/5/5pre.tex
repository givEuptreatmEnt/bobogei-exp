\documentclass[12pt, a4paper]{article}

\usepackage[hmargin=2.5cm, vmargin=2cm]{geometry}
\usepackage{amsthm, amssymb, mathtools, yhmath, graphicx}
\usepackage{fontspec, type1cm, titlesec, titling, fancyhdr, tabularx}
\usepackage{caption}
\usepackage{color}
\usepackage{hhline}
\usepackage{unicode-math}
\usepackage{nicefrac}
\usepackage[abbreviations, per-mode=symbol]{siunitx}
\usepackage{comment}
\usepackage{float}
\usepackage{subcaption}

\usepackage[CheckSingle, CJKmath]{xeCJK}
\usepackage{CJKulem}
\usepackage{enumitem}
\usepackage[usenames, dvipsnames]{xcolor}
\usepackage{colortbl}
\usepackage{circuitikz}
%\setCJKmainfont[BoldFont=cwTex Q Hei]{cwTex Q Ming}
%\setCJKsansfont[BoldFont=cwTex Q Hei]{cwTex Q Ming}
%\setCJKmonofont[BoldFont=cwTex Q Hei]{cwTex Q Ming}
\setCJKmainfont[BoldFont=cwTeX Q Hei]{cwTeX Q Ming}

\def\normalsize{\fontsize{12}{18}\selectfont}
\def\large{\fontsize{14}{21}\selectfont}
\def\Large{\fontsize{16}{24}\selectfont}
\def\LARGE{\fontsize{18}{27}\selectfont}
\def\Huge{\fontsize{20}{30}\selectfont}

%\titleformat{\section}{\bf\Large}{\arabic{section}}{24pt}{}
%\titleformat{\subsection}{\large}{\\arabic{subsection}.}{12pt}{}
\titlespacing*{\subsection}{0pt}{0pt}{1.5ex}

\parindent=24pt

\DeclarePairedDelimiter{\abs}{\lvert}{\rvert}
\DeclarePairedDelimiter{\norm}{\lVert}{\rVert}
\DeclarePairedDelimiter{\inpd}{\langle}{\rangle}
\DeclarePairedDelimiter{\ceil}{\lceil}{\rceil}
\DeclarePairedDelimiter{\floor}{\lfloor}{\rfloor}

\newcommand{\unit}[1]{\:(\text{#1})}
\newcommand{\img}{\mathsf{i}}
\newcommand{\ex}{\mathsf{e}}
\newcommand{\dD}{\mathrm{d}}
\newcommand{\dI}{\,\mathrm{d}}
\DeclareSIUnit \uF {\micro \farad}
\DeclareSIUnit \mH {\milli \henry}

\newcommand{\tri}{$\rhd$}

\title{ \bf {\huge 電子電路實驗5: Differential Amplifiers }\\ 實驗預報}
\author{B02901178 江誠敏}
%\date{2014/09/21}

\begin{document}

\maketitle

\section{Objectives}
\begin{enumerate}
  \item  To be familiar with the characteristics of differential amplifiers
  \item  To comprehend the importance of CMRR (Common Mode Rejection Ratio)
of an amplifier.  
\end{enumerate}


\section{Procedures}
\subsection{Differential Mode Small Signal Analysis}
\begin{enumerate}[itemsep=0pt]
  \item  Use 10 kΩ variable resistance for $R_{C1}, R_{C2}, R'$, and $R_S1 = R_S2 = 0$ in Fig. 5.
\item  Provide voltage source $V_{CC} = +12V$, and $V_{SS} = −6V$ to the circuit.
\item  Provide voltage source $V_{SS} = −6V$ to pin 13 of each chip of CA3046.
\item Oscilloscope  \tri Press the CH1 and CH2 MENU  \tri  Coupling  \tri  AC.
\item Oscilloscope  \tri Press the DISPLAY button  \tri  Format  \tri  YT mode.
\item Oscilloscope  \tri Press the Measure button  \tri  Observe  $V_i  (p-p)$ in CH1.
\item Use the function generator to provide the input small signal  $V_i$  and
  make sure that  $V_i  = v_{ac} sin(2 \pi ft)$, $2v_{ac} = 20 mV (p-p)$ , $f = 1 \textasciitilde 5kHz$ is
measured from the breadboard by using CH1 of oscilloscope to observe.
\item  Keep the previous adjustment of  $V_i$  constantly, and do not adjust the
amplitude tuner in function generator any further.
\item  Oscilloscope  \tri Press the Measure button  \tri  Observe $V_{O1}$ (p-p) and $V_{O2}$ (p-p)
in CH1 and CH2 at YT mode.
\item  Adjust the variable resistance of $R'$ so that voltage gain could be as high as
possible.
\item  Adjust the variable resistance of $R_{C1}$ and $R_{C2}$ so that $A_{d1}$ voltage gain could
be equal to A d2 .
\item  If  $V_o  = 0$, that is, there is no output signal, try to generate the input small
  signal  $v_i$  as  $v_i  = v_{ac} sin(2\pi ft)$, $2v_ac = 2V$ (p-p) , $f = 1 \textasciitilde 5kHz$, and repeat step
(10) ~ (11).
\item  Record $A_d$:
\item  Function generator  \tri  Press the FUNC button  \tri  Reducing Frequency and
  observe the voltage gain $A_{V_{in}}$ oscilloscope until $A_V = 0.707 A_d$ .
\item  Function generator  \tri  Press the FUNC button  \tri  Increasingly adjust the
  Frequency and observe the gain $A_{V_{in}}$ oscilloscope until $A_V = 0.707 A_d$..
\item  Record the frequency
\item  Change the frequency of small-signal input voltage, and record the input
\end{enumerate}
\subsection{Common Mode Small Signal Analysis}
\begin{enumerate}[itemsep=0pt]
  \item Keep the previous adjustment of $R_{C1}  , R_{C2} \text{ and } R'$ constantly. Use function
    generator to provide  $v_i  = v_{ac} sin(2 \pi ft)$, $2v_{ac} = \SI{1}\V$ (p-p) , $f = 1 \textasciitilde 5 kHz$.
  \item  Oscilloscope  \tri Press the Measure button  \tri  Observe $V_{O1}$ (p-p) and $V_{O2}$ (p-p)
in CH1 and CH2 at YT mode.
\item  Record $A_{cm}$
\item  Function generator  \tri  Press the FUNC button  \tri  Reducing Frequency and
  observe the voltage gain $A_{V_{in}}$ oscilloscope until $A_V \text{(Low-3dB)} = 0.707 A_{cm}$ .
\item  Function generator  \tri  Press the FUNC button  \tri  Increasingly adjust the
  Frequency and observe the gain $A_{V_{in}}$ oscilloscope until $A_V \text{(High-3dB)} = 0.707 A_{cm}$ .
\item  Record the frequency.
\item  Change the frequency of small-signal input voltage, and record the input
and output voltage shown in oscilloscope to the following table.

\end{enumerate}

\subsection{Completed Mode Small Signal Analysis}
\begin{enumerate}[itemsep=0pt]
  \item  Keep the previous adjustment of $R_{C1}$  , $R_{C2}$  and $R'$ constantly. Use the
function generator to provide:
(a)  $V_{id} = v_{ac} sin( \omega t), 2v_{ac} = 20 mV$ (p-p) , and
(b)  $V_i cm = v_{ac} sin(2 \pi ft), 2v_{ac} = \SI{1}\V$ (p-p) , $f = 1 \textasciitilde 5$ kHz.
\item  Oscilloscope  \tri Press the Measure button  \tri  Observe $V_{O1}$ (p-p) and $V_{O2}$ (p-p)
in CH1 and CH2 at YT mode.
\item  Record $A_d, A_{d1}, A_{d2}, A_{cm}, V_{O1}, V_{O2}$ 
\item  Function generator  \tri  Press the FUNC button  \tri  Increasingly adjust the
  Frequency and observe the gain $A_{V_{in}}$ oscilloscope until $A_V \text{ (High-3dB) } = 0.707  A_{cm}$ .
\item  Record the frequency.
\item  Change the frequency of small-signal input voltage, and record the input
and output voltage shown in oscilloscope to the following table.
\end{enumerate}
\end{document}


