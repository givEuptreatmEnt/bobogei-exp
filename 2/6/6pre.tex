\documentclass[12pt, a4paper]{article}

\usepackage[hmargin=2.5cm, vmargin=2cm]{geometry}
\usepackage{amsthm, amssymb, mathtools, yhmath, graphicx}
\usepackage{fontspec, type1cm, titlesec, titling, fancyhdr, tabularx}
\usepackage{caption}
\usepackage{color}
\usepackage{hhline}
\usepackage{unicode-math}
\usepackage{nicefrac}
\usepackage[abbreviations, per-mode=symbol]{siunitx}
\usepackage{comment}
\usepackage{float}
\usepackage{subcaption}

\usepackage[CheckSingle, CJKmath]{xeCJK}
\usepackage{CJKulem}
\usepackage{enumitem}
\usepackage[usenames, dvipsnames]{xcolor}
\usepackage{colortbl}
\usepackage{circuitikz}
%\setCJKmainfont[BoldFont=cwTex Q Hei]{cwTex Q Ming}
%\setCJKsansfont[BoldFont=cwTex Q Hei]{cwTex Q Ming}
%\setCJKmonofont[BoldFont=cwTex Q Hei]{cwTex Q Ming}
\setCJKmainfont[BoldFont=cwTeX Q Hei]{cwTeX Q Ming}

\def\normalsize{\fontsize{12}{18}\selectfont}
\def\large{\fontsize{14}{21}\selectfont}
\def\Large{\fontsize{16}{24}\selectfont}
\def\LARGE{\fontsize{18}{27}\selectfont}
\def\Huge{\fontsize{20}{30}\selectfont}

%\titleformat{\section}{\bf\Large}{\arabic{section}}{24pt}{}
%\titleformat{\subsection}{\large}{\\arabic{subsection}.}{12pt}{}
\titlespacing*{\subsection}{0pt}{0pt}{1.5ex}

\parindent=24pt

\DeclarePairedDelimiter{\abs}{\lvert}{\rvert}
\DeclarePairedDelimiter{\norm}{\lVert}{\rVert}
\DeclarePairedDelimiter{\inpd}{\langle}{\rangle}
\DeclarePairedDelimiter{\ceil}{\lceil}{\rceil}
\DeclarePairedDelimiter{\floor}{\lfloor}{\rfloor}

\newcommand{\unit}[1]{\:(\text{#1})}
\newcommand{\img}{\mathsf{i}}
\newcommand{\ex}{\mathsf{e}}
\newcommand{\dD}{\mathrm{d}}
\newcommand{\dI}{\,\mathrm{d}}
\DeclareSIUnit \uF {\micro \farad}
\DeclareSIUnit \mH {\milli \henry}

\newcommand{\tri}{$\rhd$}

\title{ \bf {\huge 電子電路實驗6: Power Amplifiers (PA): Class-B Output Stage }\\ 實驗預報}
\author{B02901178 江誠敏}
%\date{2014/09/21}

\begin{document}

\maketitle

\section{Objectives}
\begin{enumerate}
  \item To familiarize with the characteristics of class-B output stages of power amplifiers. 
  \item To comprehend the method of eliminating the Crossover Distortion of a class-B output stage. 
\end{enumerate}


\section{Procedures}
\subsection{Differential Mode Small Signal Analysis}
\begin{enumerate}[itemsep=0pt]
  \item  Use $\SI{100}\ohm$ cement resistance for $R_L$ in Fig 6.
  \item  Provide voltage source $V_{CC} = +15V$, and $-V_{CC} = −15V$ to the circuit.
  \item  Function generator  \tri  Press the FUNC button  \tri Set FREQ = 1 kHz, SIN
wave  \tri ATTN 0dB.
\item  Oscilloscope  \tri  Press the CH1 and CH2 MENU  \tri  Coupling  \tri  AC.
\item  Oscilloscope  \tri  Press the DISPLAY button  \tri  Format  \tri  YT mode.
\item  Oscilloscope  \tri  Press the Measure button  \tri  Observe $V_{i(p-p)}$ in CH1.
\item Using function generator to generate the input small signal $V_i$ and make sure
  that $V_i = v_{ac} \sin(2\pi ft), 2v_{ac} = 6V_{(p-p)} , f = 1 kHz$ is measured 
  from the breadboard by using CH1 of oscilloscope to observe.
\item  Keep the previous adjustment of $V_i$ constantly, and do not adjust the
amplitude tuner in function generator any further.
\item  Oscilloscope  \tri  Press the DISPLAY button  \tri  Format  \tri  XY mode.
\item Adjust VOLTS/DIV of CH1 to change the X-axis scale and of CH2 to
change the Y-axis scale, Observe whether it appears the crossover
distortion in the VTC (Voltage Transfer Curve) at the screen of
oscilloscope.
\item  If there is no crossover distortion in the VTC shown at oscilloscope or the
curve doesn’t seem to resemble the diagram shown in Fig. 2, try to
troubleshoot the bugs in your circuit. \\
※ Note: The bugs often result from the abuse of components in the circuit,
the corruption of the transistors, and incorrectly connections of the circuit.
\item  Record the value of $V_{BB}$ by measuring how many divisions the crossover distortion 
  occupies in X-axis at the screen of oscilloscope.
\end{enumerate}

\subsection{Common Mode Small Signal Analysis}
\begin{enumerate}[itemsep=0pt]
  \item Use $R_L = \SI{100}\ohm$ cement resistance, $R_E = \SI{1}\ohm, \SI{1}\kohm$ resistance for
    $R_a , R_1 , R_2$, and $\SI{1}\kohm$ variable resistance for $R_b$ , in Fig. 7.
\item  Before applying the power supplier in the circuit, MAKE SURE the
layout in your bread board is correct. Otherwise, the resistance $R_E$ will
easily be burned and corrupted because of the huge amount of the current
generated by the power BJTs.
\item  Provide voltage source $V_{CC} = +15V$, and $−V_{CC} = −15V$ to the circuit.
\item  Keep the previous adjustment of $V_i$ in step 7. constantly.
\item  Oscilloscope  \tri  Press the DISPLAY button  \tri  Format  \tri  XY mode.
\item  Adjust VOLTS/DIV of CH1 to change the X-axis scale and that of CH2
to change the Y-axis scale  \tri  Observe whether it appears the crossover
distortion in the voltage transfer curve in the screen of oscilloscope.
\item  If the curve shown in the screen of oscilloscope looks weird or ridiculous,
try to seriously get back to step (11)-(14) and find the bugs in the circuit.
\item  Adjust the $R_b$ until the crossover distortion disappears and the curve of
VTC is just a straight line.
\item  Record the value of $V_{CE3}, V_{REN}, V_{REP}, R_b, V_{BB}$
\item  Use an $\SI{8}\ohm$ speaker for $R_L$. Adjust big-signal input $V_i = v_{ac} \sin(2\pi ft)$, 
  $2v_{ac} = 2V_{(p-p)}$ .
\item  Try to identify what the different sound is as the crossover distortion is
occurring and it is just eliminated.
\item  Adjust the frequency $f$ in the function generator in $\SI{2}{\Hz} \textasciitilde \SI{20}{kHz}$ and listen
what will happen.
\item  Write them down in your homework report.
\end{enumerate}

\end{document}


