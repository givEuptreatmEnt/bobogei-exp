\documentclass[12pt, a4paper]{article}

\usepackage[hmargin=2.5cm, vmargin=2cm]{geometry}
\usepackage{amsthm, amssymb, mathtools, yhmath, graphicx}
\usepackage{fontspec, type1cm, titlesec, titling, fancyhdr, tabularx}
\usepackage{color}
\usepackage{unicode-math}
\usepackage{float}
\usepackage{hhline}
\usepackage{comment}
\usepackage[abbreviations]{siunitx}
\usepackage{csvsimple}
\usepackage{subcaption}

\usepackage[CheckSingle, CJKmath]{xeCJK}
\usepackage{CJKulem}
\usepackage{enumitem}
\usepackage{tikz}
\usepackage[siunitx]{circuitikz}
\usepackage{wrapfig}
%\setCJKmainfont[BoldFont=cwTex Q Hei]{cwTex Q Ming}
%\setCJKsansfont[BoldFont=cwTex Q Hei]{cwTex Q Ming}
%\setCJKmonofont[BoldFont=cwTex Q Hei]{cwTex Q Ming}
\setCJKmainfont[BoldFont=cwTeX Q Hei]{cwTeX Q Ming}

\def\normalsize{\fontsize{12}{18}\selectfont}
\def\large{\fontsize{14}{21}\selectfont}
\def\Large{\fontsize{16}{24}\selectfont}
\def\LARGE{\fontsize{18}{27}\selectfont}
\def\huge{\fontsize{20}{30}\selectfont}

%\titleformat{\section}{\bf\Large}{\arabic{section}}{24pt}{}
%\titleformat{\subsection}{\large}{\arabic{subsection}.}{12pt}{}
%\titlespacing*{\subsection}{0pt}{0pt}{1.5ex}

\parindent=24pt

\DeclarePairedDelimiter{\abs}{\lvert}{\rvert}
\DeclarePairedDelimiter{\norm}{\lVert}{\rVert}
\DeclarePairedDelimiter{\inpd}{\langle}{\rangle}
\DeclarePairedDelimiter{\ceil}{\lceil}{\rceil}
\DeclarePairedDelimiter{\floor}{\lfloor}{\rfloor}

\newcommand{\unit}[1]{\:(\text{#1})}
\newcommand{\df}[1]{\mathop{}\!\mathrm{d^#1}}
\newcommand{\img}{\mathrm{i}}
\newcommand{\dD}{\mathrm{d}}
\newcommand{\dI}{\,\mathrm{d}}

\title{ \bf {\Huge 電子電路實驗1:Operational Amplifier Circuits}\\ 實驗結報}
\author{B02901178 江誠敏}

\begin{document}

\maketitle


\section{實驗結果}

\subsection{Pin Voltage}
\begin{center}
  \begin{tabular}{p{2cm}p{3.5cm}}
    \hline
    Pin & Voltage \\
    \hhline{====}
    $4$ & $\SI{-15.010}\V$  \\
    $7$ & $\SI{15.013}\V$  \\
    $2$ & $\SI{0.00}\V$  \\
    $3$ & $\SI{0.00}\V$  \\
    $6$ & $\SI{0.0006}\V$  \\
    \hline
  \end{tabular} \\[10pt]
\end{center}

\subsection{Inverting configuration}
\subsubsection{ $R_2 = \SI{1}\kohm$ }
\begin{center}
  \begin{tabular}{p{3.5cm}p{2.5cm}p{2.5cm}p{2.5cm}}
    \hline
    Frequency & $v_i$ & $v_o$ & $v_o/v_i$ \\
    \hhline{====}
    $\SI{1}{\kHz}$ & $\SI{160}{\mV}$ & $\SI{1.4}{\V}$ & $8.75$  \\
    $\SI{10}{\kHz}$ & $\SI{160}{\mV}$ & $\SI{1.38}{\V}$ & $8.625$  \\
    \hline
  \end{tabular}
\end{center}

\subsubsection{ $R_2 = \SI{4.7}\kohm$ }
\begin{center}
  \begin{tabular}{p{3.5cm}p{2.5cm}p{2.5cm}p{2.5cm}}
    \hline
    Frequency & $v_i$ & $v_o$ & $v_o/v_i$ \\
    \hhline{====}
    $\SI{1}{\kHz}$ & $\SI{145}{\mV}$ & $\SI{6.88}{\V}$ & $47.49$  \\
    $\SI{10}{\kHz}$ & $\SI{148}{\mV}$ & $\SI{6.64}{\V}$ & $44.86$  \\
    \hline
  \end{tabular}
\end{center}

\subsection{Non-inverting configuration}

\subsubsection{ $R_2 = \SI{1}\kohm$ }
\begin{center}
  \begin{tabular}{p{3.5cm}p{2.5cm}p{2.5cm}p{2.5cm}}
    \hline
    Frequency & $v_i$ & $v_o$ & $v_o/v_i$ \\
    \hhline{====}
    $\SI{1}{\kHz}$ & $\SI{160}{\mV}$ & $\SI{1.681}{\V}$ & $10.51$  \\
    $\SI{10}{\kHz}$ & $\SI{160}{\mV}$ & $\SI{1.681}{\V}$ & $10.51$  \\
    \hline
  \end{tabular}
\end{center}

\subsubsection{ $R_2 = \SI{4.7}\kohm$ }
\begin{center}
  \begin{tabular}{p{3.5cm}p{2.5cm}p{2.5cm}p{2.5cm}}
    \hline
    Frequency & $v_i$ & $v_o$ & $v_o/v_i$ \\
    \hhline{====}
    $\SI{1}{\kHz}$ & $\SI{172}{\mV}$ & $\SI{7.12}{\V}$ & $41.40$  \\
    $\SI{10}{\kHz}$ & $\SI{180}{\mV}$ & $\SI{6.88}{\V}$ & $38.22$  \\
    \hline
  \end{tabular}
\end{center}

\subsection{Voltage Follower}

\begin{center}
  \begin{tabular}{p{3.5cm}p{2.5cm}p{2.5cm}p{2.5cm}}
    \hline
    Frequency & $v_i$ & $v_o$ & $v_o/v_i$ \\
    \hhline{====}
    $\SI{1}{\kHz}$ & $\SI{2.16}{\V}$ & $\SI{2.24}{\V}$ & $1.04$  \\
    $\SI{10}{\kHz}$ & $\SI{2.20}{\V}$ & $\SI{2.24}{\V}$ & $1.02$  \\
    \hline
  \end{tabular}
\end{center}


\section{心得}
新的學期開始,電子電路實驗也換了新教室。沒想到一進教室,明明都還沒有開始上課,教室裡就已人山人海
,而且有好的儀器的位置都已經被搶光了!所以我們一群人只好被流放邊疆,到儀器比較爛的位置上。
結果那個示波器還真的蠻慘的,Autoset 後波形都還會超出範圍,然後動不動就有儀器整個當機。這個故事
告訴我們第一堂課最好早一點到,不然整學期都會很慘!
\end{document}

