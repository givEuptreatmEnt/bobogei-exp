\documentclass[12pt, a4paper]{article}

\usepackage[hmargin=2.5cm, vmargin=2cm]{geometry}
\usepackage{amsthm, amssymb, mathtools, yhmath, graphicx}
\usepackage{fontspec, type1cm, titlesec, titling, fancyhdr, tabularx}
\usepackage{color}
\usepackage{unicode-math}
\usepackage{float}
\usepackage{hhline}
\usepackage{comment}
\usepackage{siunitx}
\usepackage{csvsimple}

\usepackage[CheckSingle, CJKmath]{xeCJK}
\usepackage{CJKulem}
\usepackage{enumitem}
\usepackage{tikz}
\usepackage[siunitx]{circuitikz}
\usepackage{wrapfig}
%\setCJKmainfont[BoldFont=cwTex Q Hei]{cwTex Q Ming}
%\setCJKsansfont[BoldFont=cwTex Q Hei]{cwTex Q Ming}
%\setCJKmonofont[BoldFont=cwTex Q Hei]{cwTex Q Ming}
\setCJKmainfont[BoldFont=cwTeX Q Hei]{cwTeX Q Ming}

\def\normalsize{\fontsize{12}{18}\selectfont}
\def\large{\fontsize{14}{21}\selectfont}
\def\Large{\fontsize{16}{24}\selectfont}
\def\LARGE{\fontsize{18}{27}\selectfont}
\def\huge{\fontsize{20}{30}\selectfont}

%\titleformat{\section}{\bf\Large}{\arabic{section}}{24pt}{}
%\titleformat{\subsection}{\large}{\arabic{subsection}.}{12pt}{}
%\titlespacing*{\subsection}{0pt}{0pt}{1.5ex}

\parindent=24pt

\DeclarePairedDelimiter{\abs}{\lvert}{\rvert}
\DeclarePairedDelimiter{\norm}{\lVert}{\rVert}
\DeclarePairedDelimiter{\inpd}{\langle}{\rangle}
\DeclarePairedDelimiter{\ceil}{\lceil}{\rceil}
\DeclarePairedDelimiter{\floor}{\lfloor}{\rfloor}

\newcommand{\unit}[1]{\:(\text{#1})}
\newcommand{\df}[1]{\mathop{}\!\mathrm{d^#1}}
\newcommand{\img}{\mathrm{i}}

\title{ \bf {\Huge 電子電路實驗五:RC 與 RL 電路之步級響應}\\ 實驗結報}
\author{B02901178 江誠敏}
\date{2014/10/07} 

\begin{document}

\maketitle

\section{實驗結果}
本實驗的電路圖如下:\\
\begin{center}
\begin{tikzpicture}[american voltages, scale=.8]
	\draw[color=black, thick]
	(0, 0) to [V] (0, 6) {}
	(0, 6) to [R, v=$V_R$] (6, 6) {}
	(6, 6) to [C, l=0.1<\micro\farad>, v=$V_C$] (6, 0) {}
	(6, 0) to [short] (0, 0) {}
	(3, 0) node[ground] {}
	;
\end{tikzpicture}
\quad
\begin{tikzpicture}[american voltages, scale=.8]
	\draw[color=black, thick]
	(0, 0) to [V] (0, 6) {}
	(0, 6) to [R, v=$V_R$] (6, 6) {}
	(6, 6) to [L, l=0.1<\micro\henry>, v=$V_L$] (6, 0) {}
	(6, 0) to [short] (0, 0) {}
	(3, 0) node[ground] {}
	;
\end{tikzpicture}
\end{center}


\subsection{RC電路}
量測$C = 10\si{\micro\farad}$
\begin{center}
	\begin{tabular}{p{2cm}p{3cm}p{3cm}p{2.5cm}}
	\hline
	電阻 & 時間常數測量值 & 理論值 & 相對誤差 \\
	\hline 
	\hline
	\csvreader[head to column names, late after line=\\\hline]{dt1.csv}{}%
  {\r\si{\kilo\ohm}&\ext\si{\micro s}&\trt\si{\micro s}&\er\%} \end{tabular}
\end{center}

\subsection{RL電路}
\begin{center}
	\begin{tabular}{p{2cm}p{3cm}p{3cm}p{2.5cm}}
	\hline
	電阻 & 時間常數測量值 & 理論值 & 相對誤差 \\
	\hline 
	\hline
	\csvreader[head to column names, late after line=\\\hline]{dt2.csv}{}%
  {\r\si{\kilo\ohm}&\ext\si{\micro s}&\trt\si{\micro s}&\er\%} \end{tabular}
\end{center}




\section{結報問題}

\begin{enumerate}[itemsep=20pt, topsep=10pt]
  \item {\large\bf 在做一次電路時,輸入方波的週期最好為電路時間常數(Time Constant)的三倍以上,為什麼?} \\[10pt]
    答: 如果周期不夠長,電容可能還沒有被充滿,或是電感的感應電壓尚有一定的大小,這樣我們就不能直接用量測電壓從$5 \si{V}$到$1.32 \si{V}$的時間差這個方便的方法來計算時間常數,而要用其他方法了。
  \item {\large\bf Duty Cycle的定義為何?以何鈕調整?} \\[10pt]
    答:Duty Cycle原本的定義是機器工作的時間占所有時間的比列.而對於週期性方波即是指訊號高電位占所有時間的比列。而對於我們實驗使用的SFG系列的訊號產生器(應該是吧…),調整的方法是先按下shift鍵,接者按下DUTY鍵,最後旋轉旋鈕即可調整比例。
\end{enumerate}

\section{心得}
這次的實驗電容的部分好像哪裡弄錯了,處理數據的時後才發現誤差都蠻大的(我猜是週期不夠長...)。實驗的時候我還不知道$1.32\si{\volt}$到底是怎麼來的,問了旁邊的人好像也沒人知道,看來以後還是要好好讀實驗講義。

\end{document}

