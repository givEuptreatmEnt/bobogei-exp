\documentclass[12pt, a4paper]{article}

\usepackage[hmargin=2.5cm, vmargin=2cm]{geometry}
\usepackage{amsthm, amssymb, mathtools, yhmath, graphicx}
\usepackage{fontspec, type1cm, titlesec, titling, fancyhdr, tabularx}
\usepackage{color}
\usepackage{unicode-math}
\usepackage{float}
\usepackage{hhline}
\usepackage{comment}
\usepackage{siunitx}
\usepackage{csvsimple}
\usepackage{subcaption}

\usepackage[CheckSingle, CJKmath]{xeCJK}
\usepackage{CJKulem}
\usepackage{enumitem}
\usepackage{tikz}
\usepackage[siunitx]{circuitikz}
\usepackage{wrapfig}
%\setCJKmainfont[BoldFont=cwTex Q Hei]{cwTex Q Ming}
%\setCJKsansfont[BoldFont=cwTex Q Hei]{cwTex Q Ming}
%\setCJKmonofont[BoldFont=cwTex Q Hei]{cwTex Q Ming}
\setCJKmainfont[BoldFont=cwTeX Q Hei]{cwTeX Q Ming}

\def\normalsize{\fontsize{12}{18}\selectfont}
\def\large{\fontsize{14}{21}\selectfont}
\def\Large{\fontsize{16}{24}\selectfont}
\def\LARGE{\fontsize{18}{27}\selectfont}
\def\huge{\fontsize{20}{30}\selectfont}

%\titleformat{\section}{\bf\Large}{\arabic{section}}{24pt}{}
%\titleformat{\subsection}{\large}{\arabic{subsection}.}{12pt}{}
%\titlespacing*{\subsection}{0pt}{0pt}{1.5ex}

\parindent=24pt

\DeclarePairedDelimiter{\abs}{\lvert}{\rvert}
\DeclarePairedDelimiter{\norm}{\lVert}{\rVert}
\DeclarePairedDelimiter{\inpd}{\langle}{\rangle}
\DeclarePairedDelimiter{\ceil}{\lceil}{\rceil}
\DeclarePairedDelimiter{\floor}{\lfloor}{\rfloor}

\newcommand{\unit}[1]{\:(\text{#1})}
\newcommand{\df}[1]{\mathop{}\!\mathrm{d^#1}}
\newcommand{\img}{\mathrm{i}}
\newcommand{\dD}{\mathrm{d}}
\newcommand{\dI}{\,\mathrm{d}}

\title{ \bf {\Huge 電子電路實驗9:二次線路的頻率響應}\\ 實驗結報}
\author{B02901178 江誠敏}

\begin{document}

\maketitle


\section{實驗結果}

\subsection{固定頻率,調整可變電阻}
\begin{center}
  \begin{tabular}{p{3.5cm}p{2.5cm}p{2.5cm}p{2.5cm}}
    \hline
    濾波後的頻率倍率 & $v_i$ & $v_o$ & $v_o/v_i$  \\
    \hhline{====}
    $\times 1$ & $\SI{1.02}{\V}$ & $\SI{0.898}{\V}$ & $0.880$  \\
    $\times 3$ & $\SI{960}{\mV}$ & $\SI{220}{\mV}$ & $0.229$  \\
    $\times 5$ & $\SI{960}{\mV}$ & $\SI{104}{\mV}$ & $0.108$ \\
    $\times 7$ & $\SI{960}{\mV}$ & $\SI{58.0}{\mV}$ & $0.06$  \\
    \hline
  \end{tabular} \\[10pt]
  \begin{tabular}{p{3.5cm}p{2.5cm}p{2.5cm}p{2.5cm}}
    \hline
    濾波後的頻率倍率 & 可變電阻值 & 理論電阻值 & \%誤差 \\
    \hhline{====}
    $\times 1$ & $\SI{9395}{\ohm}$ & $\SI{9739}{\ohm}$ & $-3.5 \%$\\
    $\times 3$ & $\SI{983.3}{\ohm}$ & $\SI{1082.1}{\ohm}$ & $-9.13 \%$\\
    $\times 5$ & $\SI{349}{\ohm}$ & $\SI{389.56}{\ohm}$ & $-10.41 \%$\\
    $\times 7$ & $\SI{162.5}{\ohm}$ & $\SI{198.76}{\ohm}$ & $-18.24 \%$\\
    \hline
  \end{tabular}
\end{center}

\subsection{固定可變電阻,調整頻率}
\begin{center}
  \begin{tabular}{p{3.5cm}p{2.5cm}p{2.5cm}p{2.5cm}}
    \hline
    頻率 & $v_i$ & $v_o$ & $v_o/v_i$ \\
    \hhline{====}
    $\SI{333}{\Hz}$ & $\SI{976}{\mV}$ & $\SI{424}{\V}$ & $0.434$  \\
    $\SI{200}{\Hz}$ & $\SI{1010}{\mV}$ & $\SI{320}{\mV}$ & $0.317$  \\
    $\SI{142.9}{\Hz}$ & $\SI{1020}{\mV}$ & $\SI{284}{\mV}$ & $0.278$ \\
    \hline
  \end{tabular}
\end{center}


\section{結報問題}

\begin{enumerate}[itemsep=20pt, topsep=10pt]

  \item {\large\bf 請討論三角波(Triangular Waves)的諧波分析。} \\[10pt]
    答:\\
    不妨假設三角波的波形為 \\
    \begin{minipage}{0.4\textwidth}
      \[
        f(t) = 
        \begin{dcases}
          \frac{2t}{\pi}, & \text{if } 0 \leq t \leq \frac{\pi}{2} \\
          2 - \frac{2t}{\pi}, & \text{if } \frac{\pi}{2} < t \leq \pi  \\
          -f(t), & \text{if } -\pi < t < 0 
        \end{dcases}
      \]
    \end{minipage}
    \quad
    \begin{minipage}{0.4\textwidth}
      \centering
      \begin{tikzpicture}
        \draw[->, black, thick, >=latex] (-3, 0) -- (3, 0);
        \draw[->, black, thick, >=latex] (0, -3) -- (0, 3);
        \draw[blue, very thick] (-3.14, 0) -- (-1.57, -1) -- (0, 0) -- (1.57, 1) -- (3.14, 0);
      \end{tikzpicture}
    \end{minipage} \\
    且$f(t + 2 \pi) = f(t)$,令
    \[
      f(t) = c_0 + \sum_{k = 1}^{\infty} a_k \sin(kt) + \sum_{k = 1}^{\infty} b_k \cos(kt)  
    \]
    注意到$f$是奇函數,$c_0 = 0, b_k = 0 \; \forall k$。
    \[
      \int_{-\pi}^{\pi} f(t) \sin(kt) \dI t =  \sum_{k=1}^{\infty}\int_{-\pi}^{\pi} a_k \sin(kt) \dI t = \pi a_k 
    \]
    因此
    \begin{align*}
      a_k &= \frac{1}{\pi} \int_{-\pi}^{\pi} f(t) \sin(kt) \dI t \\
          &= \frac{2}{\pi} \int_{0}^{\pi} f(t) \sin(kt) \dI t \\
          &= \frac{2}{\pi} \left( \int_{0}^{\frac{\pi}{2} } \frac{2t}{\pi}  \sin(kt) \dI t + \int_{\frac{\pi}{2} }^{\pi} \frac{2t}{\pi}  \sin(kt) \dI t \right) \\
          &= \frac{4}{\pi^2} \left( 
      \left( \frac{-t}{k} \cos(kt) + \frac{1}{k^2} \sin(kt) \right)_{0}^{\pi/2} + 
      \left( \frac{-\pi}{k} \cos(kt) + \frac{t}{k} \cos(kt) - \frac{1}{k^2} \sin(kt) \right)_{0}^{\pi/2}
    \right) \\
    &= \frac{4}{\pi^2} \frac{2 \sin(k \pi / 2)}{k^2}  \\
    &=  \begin{dcases}
    0, & \text{if } k = 2n \\
    \frac{8}{\pi^2 k^2} , & \text{if } k = 4n + 1 \\
    \frac{-8}{\pi^2 k^2} , & \text{if } k = 4n + 3 \\
  \end{dcases}
\end{align*}
得出
\[
  f(t) = \sum_{n=0}^{\infty} \frac{(-1)^{n} \cdot 8}{\pi^2 (2n+1)^2} 
\]


\end{enumerate}

\section{心得}
這次是這個學期的最後一個實驗了,其實我是蠻喜歡做實驗的,可以玩很多有趣的儀器。希望下學期可以使用更多沒有玩過的儀器或是元件!
\end{document}

