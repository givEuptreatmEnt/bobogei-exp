\documentclass[12pt, a4paper]{article}

\usepackage[hmargin=2.5cm, vmargin=2cm]{geometry}
\usepackage{amsthm, amssymb, mathtools, yhmath, graphicx}
\usepackage{fontspec, type1cm, titlesec, titling, fancyhdr, tabularx}
\usepackage{caption}
\usepackage{color}
\usepackage{hhline}
\usepackage{unicode-math}
\usepackage{nicefrac}
\usepackage[abbreviations]{siunitx}
\usepackage{comment}
\usepackage{float}

\usepackage[CheckSingle, CJKmath]{xeCJK}
\usepackage{CJKulem}
\usepackage{enumitem}
\usepackage[usenames, dvipsnames]{xcolor}
\usepackage{colortbl}
\usepackage{circuitikz}
%\setCJKmainfont[BoldFont=cwTex Q Hei]{cwTex Q Ming}
%\setCJKsansfont[BoldFont=cwTex Q Hei]{cwTex Q Ming}
%\setCJKmonofont[BoldFont=cwTex Q Hei]{cwTex Q Ming}
\setCJKmainfont[BoldFont=cwTeX Q Hei]{cwTeX Q Ming}

\def\normalsize{\fontsize{12}{18}\selectfont}
\def\large{\fontsize{14}{21}\selectfont}
\def\Large{\fontsize{16}{24}\selectfont}
\def\LARGE{\fontsize{18}{27}\selectfont}
\def\Huge{\fontsize{20}{30}\selectfont}

\titleformat{\section}{\bf\Large}{\arabic{section}}{24pt}{}
\titleformat{\subsection}{\large}{\arabic{subsection}.}{12pt}{}
\titlespacing*{\subsection}{0pt}{0pt}{1.5ex}

\parindent=24pt

\DeclarePairedDelimiter{\abs}{\lvert}{\rvert}
\DeclarePairedDelimiter{\norm}{\lVert}{\rVert}
\DeclarePairedDelimiter{\inpd}{\langle}{\rangle}
\DeclarePairedDelimiter{\ceil}{\lceil}{\rceil}
\DeclarePairedDelimiter{\floor}{\lfloor}{\rfloor}

\newcommand{\unit}[1]{\:(\text{#1})}
\newcommand{\img}{\mathsf{i}}
\newcommand{\ex}{\mathsf{e}}
\newcommand{\dD}{\mathrm{d}}
\newcommand{\dI}{\,\mathrm{d}}
\DeclareSIUnit \uF {\micro \farad}
\DeclareSIUnit \mH {\milli \henry}

\title{ \bf {\huge 電子電路實驗7:雙極非線性元件特性曲線之簡單測量}\\ 實驗預報}
\author{B02901178 江誠敏}
%\date{2014/09/21}

\begin{document}

\maketitle

\section{實驗目的}
\begin{enumerate}[itemsep=0pt]
  \item 學習如何使用示波器的 X-Y 模式(X-Y mode)。
  \item 利用示波器測量一些雙極非線性元件之特性曲線。
  \item 同軸信號線(cable)、探針(probe)之認識及使用。
\end{enumerate}


\section{實驗步驟}
將示波器之操作模式設於 X-Y mode。分別以矽二極體、鍺二極體、齊納二極體,以及$\SI{5.1}\kohm, \SI{100}\ohm$電阻,進行下列步驟:
\begin{enumerate}[itemsep=0pt]
  \item 將非線性元件如圖 7.1 放於非線性元件位置(注意極性),電阻放於圖中電阻位置,並適當 地調整兩信號輸入端的增益(gain factor,即調 V/D)鈕,即可由示波器得出該元件之特性
    ve)。但此時 Y 軸曲線相反,請調整成為 Y 軸方向正確的方式。
  \item 改變輸入信號的頻率(frequency),並記錄高頻與低頻的非線性元件特性曲線。
  \item 改變輸入信號的振幅(amplitude),並記錄該元件的特性曲線有何變化。注意觀察特性曲線時,亦應注意其反向偏壓(reverse bias)的變化。
\end{enumerate}

\begin{figure}[H]
  \centering
  \begin{tikzpicture}[american voltages, scale=0.8]
    \draw[color=black, thick]
    (0, 0) to [V=$V_S$] (0, 6) 
    (0, 6) to [R] (6, 6) 
    (6, 6) to [D] (6, 0)
    (6, 0) to [short] (0, 0)
    (3, 0) node[ground]{}
    ;
  \end{tikzpicture}
\end{figure}



\section{預報問題}

\begin{enumerate}[itemsep=20pt, topsep=10pt]
  \item {\large\bf 如何將示波器之操作模式設定於 X-Y mode?} \\[10pt]
    跟據我的記憶以及參閱類似型號的manual,似乎是按Display>Format>將V-t mode 改成 X-Y mode. 在以前利薩如圖形時有使用過一次。
  \item {\large\bf 實驗步驟 2 中,使用示波器的何種功能可將示波器螢幕上方向相反的 Y 軸扳正?
    } \\[10pt]
    可在Math>Invert ChX 中將Channel X的訊號取反號。

\end{enumerate}

\end{document}


