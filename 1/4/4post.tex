\documentclass[12pt, a4paper]{article}

\usepackage[hmargin=2.5cm, vmargin=2cm]{geometry}
\usepackage{amsthm, amssymb, mathtools, yhmath, graphicx}
\usepackage{fontspec, type1cm, titlesec, titling, fancyhdr, tabularx}
\usepackage{color}
\usepackage{unicode-math}
\usepackage{float}
\usepackage{hhline}
\usepackage{comment}
\usepackage{siunitx}
\usepackage{csvsimple}

\usepackage[CheckSingle, CJKmath]{xeCJK}
\usepackage{CJKulem}
\usepackage{enumitem}
\usepackage{tikz}
\usepackage[siunitx]{circuitikz}
\usepackage{wrapfig}
%\setCJKmainfont[BoldFont=cwTex Q Hei]{cwTex Q Ming}
%\setCJKsansfont[BoldFont=cwTex Q Hei]{cwTex Q Ming}
%\setCJKmonofont[BoldFont=cwTex Q Hei]{cwTex Q Ming}
\setCJKmainfont[BoldFont=cwTeX Q Hei]{cwTeX Q Ming}

\def\normalsize{\fontsize{12}{18}\selectfont}
\def\large{\fontsize{14}{21}\selectfont}
\def\Large{\fontsize{16}{24}\selectfont}
\def\LARGE{\fontsize{18}{27}\selectfont}
\def\huge{\fontsize{20}{30}\selectfont}

%\titleformat{\section}{\bf\Large}{\arabic{section}}{24pt}{}
%\titleformat{\subsection}{\large}{\arabic{subsection}.}{12pt}{}
%\titlespacing*{\subsection}{0pt}{0pt}{1.5ex}

\parindent=24pt

\DeclarePairedDelimiter{\abs}{\lvert}{\rvert}
\DeclarePairedDelimiter{\norm}{\lVert}{\rVert}
\DeclarePairedDelimiter{\inpd}{\langle}{\rangle}
\DeclarePairedDelimiter{\ceil}{\lceil}{\rceil}
\DeclarePairedDelimiter{\floor}{\lfloor}{\rfloor}

\newcommand{\unit}[1]{\:(\text{#1})}
\newcommand{\df}[1]{\mathop{}\!\mathrm{d^#1}}
\newcommand{\img}{\mathrm{i}}

\title{ \bf {\Huge 電子電路實驗四:相位測量}\\ 實驗結報}
\author{B02901178 江誠敏}
\date{2014/10/07} 

\begin{document}

\maketitle

\section{實驗結果}
本實驗的電路圖如下:\\
\begin{center}
\begin{tikzpicture}[american voltages, scale=.8]
	\draw[color=black, thick]
	(0, 0) to [V] (0, 6) {}
	(0, 6) to [R, l=1<\kilo\ohm>, v=$V_R$] (6, 6) {}
	(6, 6) to [C, l=0.1<\micro\farad>, v=$V_C$] (6, 0) {}
	(6, 0) to [short] (0, 0) {}
	(3, 0) node[ground] {}
	;
\end{tikzpicture}
\end{center}


\subsection{利薩如圖形法}
\begin{center}
	\begin{tabular}{p{2cm}p{2cm}p{2cm}p{2.5cm}p{2.5cm}p{2cm}}
	\hline
	頻率 & $y$極值 & $y$截距 & 量測相位差 & 理論值 & 相對誤差 \\
	\hline
	\hline
	\csvreader[head to column names, late after line=\\\hline]{tab1.csv}{}%
	{\cola&\colb&\colc&\ang{\cold}&\ang{\cole}&\colf\%}
\end{tabular}
\end{center}


\subsection{雙軌跡直接測量法}

\begin{center}
	\begin{tabular}{p{2cm}p{2cm}p{2cm}p{2.5cm}p{2.5cm}p{2cm}}
	\hline
	頻率 & 時間差 & 量測相位差 & 理論值 & 相對誤差 \\
	\hline
	\hline
	\csvreader[head to column names, late after line=\\\hline]{tab2.csv}{}%
	{\cola&\colb\:(\si{\micro\second})&\ang{\colc}&\ang{\cold}&\cole\%}
\end{tabular}
\end{center}


\section{結報問題}

\begin{enumerate}[itemsep=20pt, topsep=10pt]
	\item {\large\bf 當 X-Y mode 時,Lissajous Figures Method 圖形:} 
		\begin{enumerate}[label=(\alph*)]
			\item {\large\bf 試述軌跡方向與相位差之關係?} \\[5pt]
		答: 令$\Delta \phi = \phi_x - \phi_y$,也就是說如果$\Delta \phi> 0$,$x$領先$y$,則$x$會先達到最大值,接者$y$才會,因此軌跡方向會是逆時鐘方向,反之如果$y$領先$x$,則軌跡以順時針方向轉動。
			\item {\large\bf 什麼樣的情況會造成圖形不成封閉曲線?}\\[5pt]
				答: 圖形封閉的話有一點會在兩個時間被經過。假設$x(t) = \sin(\omega_x t + \phi_x) , y(t) = \sin(\omega_y t + \phi_y)$,因此如果在時間$t_1, t_2$時在同一點,則$\Delta t = t_1 - t_2$必需要是$2 \pi / \omega_x, 2 \pi / \omega_y$的整數倍,因此$\omega_x / \omega_y$必需是有理數,並且如果$\omega_x / \omega_y = a/b$,取$\Delta t = 2 \pi a / \omega_x$即可,因此圖形不成封閉曲線若且唯若兩頻率比不為有理數。
			\item {\large\bf 什麼樣的情況會造成圖形出現有交叉點?}\\[5pt]
				答: 如果頻率相等的時後顯然圖形是一個橢圓,不會有交叉點。現在不失一般性假設$\omega_x > \omega_y$,且$x(t) = \cos(\omega_x t +\phi), y(t) = \cos(\omega_y t)$,取$ t_1 = \pi / \omega_x, t_2 = -\pi / \omega_x$,可以知道$x(t_1) = x(t_2)$因為這兩點的相位是$2 \pi \omega_x / \omega_x = 2 \pi$,而$y(t_1) = \cos(\omega t1) = \cos(\omega -t1) = \cos(\omega t2) = y(t_2)$,但$y'(t_1) \neq y'(t_2)$,因此這點是個交叉點而非下個周期同相位的點。總結以上只要頻率不相等,就會有交叉點。
		\end{enumerate}
	\item {\large\bf 如何由 Dual-Trace Method 看出螢幕上兩波形為超前或落後(Lead/Lag)?}\\[10pt]
		答:取兩訊號相鄰的兩個波峰,容易知道較左側的在比較早的時間就達到波峰,因此左側的領先。
	\item {\large\bf 試述當電容改成電感時有何差異?}
		答:電容換成電感時,阻抗從$\dfrac{1}{\img \omega C} \rightarrow \img \omega L$,因此電感的電壓會領先,並且隨著$\omega$上升,電感對電源電壓的相位差下降。
		
	\item {\large\bf 根據結果,Phase Measurement 以何方法為佳?}
		答:以誤差來看,顯然Dual Trace Method較好一些。而我認為跟實驗的儀器有關,因為本實驗的示波器在XY Mode下不能使用Cursor測量,只能用肉眼估計,造成不小的誤差。
	\item {\large\bf 信號產生器的 DC Offset 之用法為何?又 Attenuator 呢?}
		答:DC Offect會給交流訊號一個直流的偏移,比如說本來交流訊號是從$-5\si{\volt} \sim +5\si{\volt}$,如果DC Offset調至$2 \si{\volt}$則輸出訊號會變成$-3\si{\volt} \sim +7 \si{\volt}$。而Attenuator則是衰減器,如果被打開後訊號將會被衰減成$10$倍。
\end{enumerate}

\section{心得}
這次的實驗還算簡單,只要線路不要接錯應該都可以做蠻快的。只是一開始打開示波器的時後別人的menu都有XY Mode,我的居然沒有,大概是前一個人有用其他選單(似乎會停在最後一個使用的menu上),好在把儀器上的按鈕全按一輪終於按到了。

\end{document}

